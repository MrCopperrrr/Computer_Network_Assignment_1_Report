\part*{Chương 6: Kết luận}
\addcontentsline{toc}{part}{Chương 6: Kết luận}
%============================================================================

\section*{6.1. Tổng kết kết quả đạt được}
\addcontentsline{toc}{section}{6.1. Tổng kết kết quả đạt được}

Sau quá trình phân tích, thiết kế và hiện thực, nhóm đã hoàn thành đầy đủ các mục tiêu đề ra, xây dựng thành công một ứng dụng mạng hoàn chỉnh, qua đó áp dụng hiệu quả các kiến thức nền tảng về mạng máy tính, giao thức truyền thông và lập trình socket.

\begin{itemize}
    \item \textbf{Hoàn thiện HTTP Server và cơ chế Cookie:}  
    Hệ thống đã hiện thực thành công máy chủ HTTP và Proxy ở mức socket thấp, đảm bảo xử lý chính xác luồng xác thực và quản lý phiên làm việc (session) thông qua cookie. Chức năng này đáp ứng đầy đủ yêu cầu của Nhiệm vụ 2.1.

    \item \textbf{Xây dựng kiến trúc Chat Hybrid:}  
    Dự án đã triển khai một mô hình chat kết hợp (hybrid) giữa hai kiến trúc Client–Server và Peer–to–Peer, bao gồm hai thành phần chính là Tracker Server (điều phối và khám phá) và Peer Client (giao tiếp trực tiếp). Hệ thống hoạt động ổn định, có khả năng mở rộng linh hoạt.

    \item \textbf{Hiện thực các hình thức giao tiếp đa dạng:}  
    Ứng dụng hỗ trợ ba chế độ liên lạc chính: chat nhóm (qua kênh \#general), chat riêng P2P, và broadcast toàn hệ thống, đáp ứng đầy đủ nhu cầu trao đổi giữa các người dùng.

    \item \textbf{Phát triển giao diện Web hoàn chỉnh:}  
    Giao diện người dùng được xây dựng bằng HTML5, CSS3 và JavaScript (ES6), kết nối trực tiếp với backend Python thông qua WebSocket, mang lại trải nghiệm tương tác thời gian thực, mượt mà và hiện đại.

    \item \textbf{Tối ưu trải nghiệm và tính tiện dụng:}  
    Hệ thống hỗ trợ lưu trạng thái đăng nhập bằng cơ chế \textit{localStorage}, tự động đăng nhập lại sau khi tải lại trang, đồng thời đảm bảo sự ổn định trong quá trình thử nghiệm thực tế.
\end{itemize}

Tổng thể, đề tài đã được hoàn thiện và vận hành đúng theo mục tiêu ban đầu, vượt qua các yêu cầu cốt lõi về kỹ thuật và chức năng.

%============================================================================
\section*{6.2. Hạn chế của hệ thống}
\addcontentsline{toc}{section}{6.2. Hạn chế của hệ thống}

Mặc dù hệ thống đã hoạt động ổn định, một số hạn chế vẫn tồn tại, chủ yếu ở khía cạnh bảo mật và khả năng mở rộng:

\begin{itemize}
    \item \textbf{Bảo mật:}  
    Các gói tin và thông tin người dùng (bao gồm mật khẩu và nội dung tin nhắn) hiện vẫn được truyền ở dạng văn bản thô, chưa được mã hóa. Việc lưu trữ mật khẩu trực tiếp trên Tracker Server cũng tiềm ẩn rủi ro an ninh.

    \item \textbf{Khả năng chịu lỗi và mở rộng:}  
    Tracker Server đang là \textit{Single Point of Failure}. Nếu máy chủ này gặp sự cố, toàn bộ mạng lưới sẽ không thể khám phá hoặc kết nối với nhau.

    \item \textbf{Độ tin cậy của giao thức P2P:}  
    Cơ chế truyền tin giữa các peer còn đơn giản, chưa có biện pháp xử lý mất gói, xác nhận tin nhắn hay tái truyền dữ liệu khi xảy ra lỗi mạng.

    \item \textbf{Vấn đề NAT Traversal:}  
    Hệ thống hiện chỉ hoạt động ổn định trong mạng nội bộ (LAN). Khi các peer nằm sau các bộ định tuyến NAT khác nhau, kết nối P2P trực tiếp sẽ thất bại do chưa tích hợp các kỹ thuật vượt NAT như STUN/TURN.
\end{itemize}

%============================================================================
\section*{6.3. Hướng phát triển}
\addcontentsline{toc}{section}{6.3. Hướng phát triển}

Dựa trên những hạn chế trên, nhóm đề xuất một số hướng mở rộng và cải tiến cho hệ thống trong tương lai:

\begin{itemize}
    \item \textbf{Tăng cường bảo mật:}  
    \begin{itemize}
        \item \textit{Mã hóa đầu cuối (End-to-End Encryption):} Áp dụng các thuật toán như RSA hoặc AES để mã hóa nội dung tin nhắn ngay tại phía người gửi, đảm bảo chỉ người nhận có thể giải mã.  
        \item \textit{Băm mật khẩu (Password Hashing):} Sử dụng các thuật toán như bcrypt hoặc Argon2 để lưu trữ mật khẩu an toàn thay vì dạng văn bản thuần.
    \end{itemize}

    \item \textbf{Nâng cao khả năng mở rộng và tin cậy:}  
    \begin{itemize}
        \item Triển khai mô hình \textit{Multi-Tracker} hoặc phân tán bằng \textit{Distributed Hash Table (DHT)} để loại bỏ sự phụ thuộc vào một server trung tâm.
        \item Bổ sung cơ chế đồng bộ và xác nhận tin nhắn (ACK), đảm bảo độ tin cậy khi truyền dữ liệu qua mạng không ổn định.
    \end{itemize}

    \item \textbf{Cải thiện trải nghiệm người dùng:}  
    \begin{itemize}
        \item Bổ sung chức năng gửi tập tin (File Transfer) qua giao thức P2P.  
        \item Hiển thị trạng thái người dùng (online/offline/typing) và cho phép tùy chỉnh hồ sơ cá nhân.  
        \item Mở rộng hệ thống thông báo và danh sách bạn bè, hướng tới trải nghiệm tương tự các ứng dụng chat chuyên nghiệp.
    \end{itemize}
\end{itemize}

\noindent
\textbf{Kết luận chung:}  
Đề tài không chỉ giúp nhóm củng cố kiến thức lý thuyết về mạng máy tính mà còn rèn luyện kỹ năng lập trình mạng ở cấp độ hệ thống. Kết quả đạt được chứng minh tính khả thi của việc xây dựng các ứng dụng mạng phân tán từ nền tảng socket, mở ra nhiều hướng phát triển chuyên sâu hơn trong tương lai.

