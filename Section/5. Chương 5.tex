\part*{Chương 5: Kiểm thử và kết quả}
\addcontentsline{toc}{part}{Chương 5: Kiểm thử và kết quả}
%============================================================================

\section*{5.1. Kịch bản và Môi trường kiểm thử}
\addcontentsline{toc}{section}{5.1. Kịch bản và Môi trường kiểm thử}
\begin{itemize}
    \item \textbf{Môi trường: }Các kịch bản được thực hiện trên một máy tính duy nhất chạy hệ điều hành Windows, mô phỏng một môi trường mạng cục bộ (LAN). Các tiến trình server và client được khởi chạy trên các cửa sổ dòng lệnh (PowerShell) riêng biệt. Giao diện người dùng được truy cập thông qua trình duyệt web ở chế độ ẩn danh để đảm bảo không bị ảnh hưởng bởi cache.
    \item \textbf{Địa chỉ IP: }
    \begin{itemize}
        \item Tracker Server chạy trên \texttt{0.0.0.0:8001}.
        \item Các Peer Client kết nối đến Tracker Server qua địa chỉ \texttt{127.0.0.1:8001}.
        \item Địa chỉ P2P được các Peer Client đăng ký với Tracker là địa chỉ IP cục bộ của máy (ví dụ: \texttt{192.168.1.6}).
    \end{itemize}
\end{itemize}
%============================================================================
\section*{5.2. Kết quả thực nghiệm}
\addcontentsline{toc}{section}{5.2. Kết quả thực nghiệm}
\subsection*{5.2.1. Kiểm thử HTTP Server và Cookie Session bằng giao diện UI/UX}
Kịch bản này nhằm kiểm tra khả năng xử lý \textbf{HTTP request}, \textbf{xác thực}, và \textbf{quản lý phiên bằng cookie} của máy chủ HTTP được tự xây dựng từ đầu.
\subsubsection*{Bước 1: Khởi chạy các Server}

\begin{itemize}

    \item \textbf{Mở Terminal 1}, khởi chạy \textit{Backend Server}:
    \begin{lstlisting}[language=bash]
python start_backend.py --server-ip <local-ip> --server-port 9000
    \end{lstlisting}
    \item \textbf{Kết quả: }Log từ PowerShell
    \begin{lstlisting}[language=bash]
[Backend] Listening on port 9000    
    \end{lstlisting}
    \item \textbf{Mở Terminal 2}, khởi chạy \textit{Proxy Server}:
    \begin{lstlisting}[language=bash]
python start_proxy.py --server-ip 0.0.0.0 --server-port 8080
    \end{lstlisting}
    \item \textbf{Kết quả: }Log từ PowerShell
    \begin{lstlisting}[language=bash]
[Host] 192.168.1.6:8080
[PORT] 192.168.1.6:9000
[POLICY] round-robin
[Proxy] Listening on IP 0.0.0.0 port 8080    
    \end{lstlisting}
\end{itemize}

\subsubsection*{Bước 2: Kiểm thử luồng xác thực trên trình duyệt}

\begin{itemize}
    \item Mở trình duyệt ở chế độ ẩn danh, truy cập địa chỉ:
    \begin{center}
        \texttt{http://<local-ip>:8080/}
    \end{center}
    \textbf{Kết quả:} Trang web trả về lỗi \texttt{401 Unauthorized} vì chưa có cookie xác thực.
    \begin{figure}[H]
    \centering
    \setlength{\fboxsep}{2pt}     
    \setlength{\fboxrule}{0.5pt}   
    \fbox{\includegraphics[scale=0.4]{Picture/401.png}}
    \caption{Giao diện trang web trả về lỗi 401 Unauthorized}
    \end{figure}

    \item Truy cập trang đăng nhập:
    \begin{center}
        \texttt{http://<local-ip>:8080/login.html}
    \end{center}
    \textbf{Kết quả:} Form đăng nhập hiển thị chính xác.
    \begin{figure}[H]
    \centering
    \setlength{\fboxsep}{2pt}     
    \setlength{\fboxrule}{0.5pt}   
    \fbox{\includegraphics[scale=0.4]{Picture/login.png}}
    \caption{Giao diện trang web hiển thị form đăng nhập}
    \end{figure}

    \item Nhập thông tin:
    \begin{center}
        \texttt{username = admin}, \quad \texttt{password = password}
    \end{center}
    Sau đó gửi form.

    \textbf{Kết quả:}
    \begin{itemize}
        \item Đăng nhập thành công, trình duyệt được chuyển hướng đến \texttt{index.html}.
        \item Trong Developer Tools (F12), một cookie \texttt{auth=true} được thiết lập.
    \end{itemize}
    \begin{figure}[H]
    \centering
    \setlength{\fboxsep}{2pt}     
    \setlength{\fboxrule}{0.5pt}   
    \fbox{\includegraphics[scale=0.4]{Picture/loginok.png}}
    \caption{Giao diện trang web sau khi đăng nhập thành công và có cookie cho phiên đăng nhập}
    \end{figure}

    \item Truy cập lại \texttt{http://<local-ip>:8080/}.

    \textbf{Kết quả:} Trang \texttt{index.html} được hiển thị vì trình duyệt đã gửi kèm cookie hợp lệ.
    \begin{figure}[H]
        \centering
        \setlength{\fboxsep}{2pt}     
        \setlength{\fboxrule}{0.5pt}   
        \fbox{\includegraphics[scale=0.4]{Picture/loginok.png}}
        \caption{Giao diện trang web sau khi reload lại}
    \end{figure}
\end{itemize}


\subsubsection*{Kết luận:}  
Chức năng xử lý request, xác thực đăng nhập và quản lý phiên bằng cookie hoạt động chính xác theo thiết kế. Hệ thống có thể phục vụ nhiều client đồng thời và duy trì phiên làm việc ổn định.

% -------------------------------------------------------------
\subsection*{5.2.2. Kiểm thử HTTP Server và Cookie Session bằng phần mềm Postman}
Sử dụng ứng dụng Postman Desktop để kiểm tra luồng xác thực cookie.
\subsubsection*{Bước 1: Khởi chạy các Server}
\begin{itemize}
    \item \textbf{Mở Terminal 1}, khởi chạy \textit{Backend Server}:
    \begin{lstlisting}[language=bash]
python start_backend.py --server-ip <local-ip> --server-port 9000
    \end{lstlisting}
    \item \textbf{Kết quả: }Log từ PowerShell
    \begin{lstlisting}[language=bash]
[Backend] Listening on port 9000    
    \end{lstlisting}
    \item \textbf{Mở Terminal 2}, khởi chạy \textit{Proxy Server}:
    \begin{lstlisting}[language=bash]
python start_proxy.py --server-ip 0.0.0.0 --server-port 8080
    \end{lstlisting}
    \item \textbf{Kết quả: }Log từ PowerShell
    \begin{lstlisting}[language=bash]
[Host] 192.168.1.6:8080
[PORT] 192.168.1.6:9000
[POLICY] round-robin
[Proxy] Listening on IP 0.0.0.0 port 8080    
    \end{lstlisting}
\end{itemize}

\subsubsection*{Bước 2: Kiểm tra bằng cách gọi các API}
\begin{itemize}
    \item \textbf{Truy cập vào trang được bảo vệ (không Cookie):}
    \begin{itemize}
        \item \textbf{Phương thức: }GET
        \item \textbf{URL: } http://<local-ip>:8080/
        \item Nhấn \textbf{Send}
        \item \textbf{Kết quả: }Status code \texttt{401 Unauthorized}
    \begin{figure}[H]
        \centering
        \setlength{\fboxsep}{2pt}     
        \setlength{\fboxrule}{0.5pt}   
        \fbox{\includegraphics[scale=0.5]{Picture/401PM.png}}
        \caption{Giao diện Postman khi truy cập trang web được bảo vệ mà không có cookie}
    \end{figure}
    \end{itemize}

    \item \textbf{Đăng nhập để nhận được Cookie:}
    \begin{itemize}
        \item \textbf{Phương thức: }POST
        \item \textbf{URL: } http://<local-ip>:8080/login
        \item Chuyển sang tab \textbf{Body} và chọn \texttt{x-www-form-urlencoded}
        \item Nhập các cặp key-value sau:
        \begin{verbatim}
            `username`: `admin`
            `password`: `password`
        \end{verbatim}
        \item Nhấn \textbf{Send}
        \item \textbf{Kết quả: }Status code \texttt{200 OK}. Trong tab \texttt{Cookie} của Response, sẽ thấy Cookie \texttt{auth=true}. Postman sẽ tự động lưu và sử dụng cookie này cho các request tiếp theo đến ip này.
    \begin{figure}[H]
        \centering
        \setlength{\fboxsep}{2pt}     
        \setlength{\fboxrule}{0.5pt}   
        \fbox{\includegraphics[scale=0.5]{Picture/loginokPM.png}}
        \caption{Giao diện Postman khi truy cập trang web thành công}
    \end{figure}
    \begin{figure}[H]
        \centering
        \setlength{\fboxsep}{2pt}     
        \setlength{\fboxrule}{0.5pt}   
        \fbox{\includegraphics[scale=0.5]{Picture/loginokcookiePM.png}}
        \caption{Giao diện Postman khi truy cập thành công cookie đã thiết lập }
    \end{figure}
    \end{itemize}

    \item \textbf{Truy cập vào trang được bảo vệ (có Cookie):}
    \begin{itemize}
        \item \textbf{Phương thức: }GET
        \item \textbf{URL: } http://<local-ip>:8080/
        \item Nhấn \textbf{Send}
        \item \textbf{Kết quả: }Status code \texttt{200 OK}. Body của response sẽ chứa nội dung của trang \texttt{index.htm}
    \begin{figure}[H]
        \centering
        \setlength{\fboxsep}{2pt}     
        \setlength{\fboxrule}{0.5pt}   
        \fbox{\includegraphics[scale=0.5]{Picture/reloadPM.png}}
        \caption{Giao diện Postman khi truy cập trang web được bảo vệ có cookie}
    \end{figure}
    \end{itemize}
\end{itemize}


% -------------------------------------------------------------
\subsection*{5.2.3. Kiểm thử Ứng dụng Chat Hybrid bằng UI/UX}

Kịch bản này kiểm thử toàn bộ luồng hoạt động của \textbf{ứng dụng chat hybrid}, từ quá trình khám phá peer đến giao tiếp \textbf{P2P (Peer-to-Peer)} và quản lý session phía client.

\subsubsection*{Bước 1: Khởi chạy Tracker Server và các Peer Clients}
\begin{itemize}
    \item \textbf{Mở Terminal 1: }Khởi chạy Tracker Server:
    \begin{lstlisting}[language=bash]
    python start_chatapp.py --server-ip 0.0.0.0 --server-port 8001
    \end{lstlisting}
    \textbf{Kết quả:} Server khởi động thành công và lắng nghe trên port 8001.
    \begin{lstlisting}[language=bash]
    ============================================================
    Starting Chat Tracker Server
    IP: 0.0.0.0
    Port: 8001
    ============================================================
    [Backend] Listening on port 8001
    \end{lstlisting}
    \item \textbf{Mở Terminal 2: }Khởi chạy Peer Alice:
    \begin{lstlisting}[language=bash]
    py peer_client.py --username alice --peer-port 9101 --tracker-ip 
    127.0.0.1 --tracker-port 8001
    \end{lstlisting}
    \textbf{Kết quả:} Peer Alice khởi chạy thành công 
    \begin{lstlisting}[language=bash]
    [INIT] Peer 'alice' configured.
    [INIT] P2P Port: 9101
    [INIT] HTTP UI Port: 19101
    [INIT] WebSocket UI Port: 29101
    [P2P Server] Listening for peers on 0.0.0.0:9101

    ============================================================
    [HTTP Server] Serving custom UI at http://localhost:19101/
    Peer Client is running.
    Open http://localhost:19101/www/chat.html in a browser to use the UI.  
    Press Ctrl+C to stop.
    ============================================================

    [UI Server] WebSocket server running on localhost:29101
    127.0.0.1 - - [12/Nov/2025 17:20:41] "GET / HTTP/1.1" 200
    \end{lstlisting}
    \item \textbf{Mở Terminal 3: }Khởi chạy Peer Bob:
    \begin{lstlisting}[language=bash]
    py peer_client.py --username bob --peer-port 9102 --tracker-ip 
    127.0.0.1 --tracker-port 8001
    \end{lstlisting}
    \textbf{Kết quả:} Peer Alice khởi chạy thành công 
    \begin{lstlisting}[language=bash]
    [INIT] Peer 'bob' configured.
    [INIT] P2P Port: 9102
    [INIT] HTTP UI Port: 19102
    [INIT] WebSocket UI Port: 29102
    [P2P Server] Listening for peers on 0.0.0.0:9102

    ============================================================
    Peer Client is running.
    Open http://localhost:19102/www/chat.html in a browser to use the UI.  
    Press Ctrl+C to stop.
    ============================================================
    
    [HTTP Server] Serving custom UI at http://localhost:19102/
    [UI Server] WebSocket server running on localhost:29102
    \end{lstlisting} 
    \item \textbf{Mở Terminal 4: }Khởi chạy Peer John:
    \begin{lstlisting}[language=bash]
    py peer_client.py --username john --peer-port 9103 --tracker-ip 
    127.0.0.1 --tracker-port 8001
    \end{lstlisting}
    \textbf{Kết quả:} Peer Alice khởi chạy thành công 
    \begin{lstlisting}[language=bash]
    [INIT] Peer 'john' configured.
    [INIT] P2P Port: 9103
    [INIT] HTTP UI Port: 19103
    [INIT] WebSocket UI Port: 29103
    [P2P Server] Listening for peers on 0.0.0.0:9103

    ============================================================
    Peer Client is running.
    Open http://localhost:19103/www/chat.html in a browser to use the UI.  
    Press Ctrl+C to stop.
    ============================================================

    [HTTP Server] Serving custom UI at http://localhost:19103/
    [UI Server] WebSocket server running on localhost:29103
    \end{lstlisting}
    
\end{itemize}
\subsubsection*{Bước 3: Kiểm thử truy cập và đăng nhập trên trình duyệt}
\begin{itemize}
    \item Mở trình duyệt ở chế độ ẩn danh, truy cập địa chỉ:
    \begin{itemize}
        \item Alice:
        \begin{center}
        \texttt{http://localhost:19101/www/chat.html}
        \end{center}
        \item Bob:
        \begin{center}
        \texttt{http://localhost:19102/www/chat.html}
        \end{center}
        \item John:
        \begin{center}
        \texttt{http://localhost:19103/www/chat.html}
        \end{center}
    \end{itemize}
    \textbf{Kết quả: } Trang web trả về 3 giao diện đăng nhập cho cả 3 tài khoản Alice, Bob, John đã được điền sẵn các thông tin như Username và P2P port.
    \begin{figure}[H]
        \centering
        \setlength{\fboxsep}{2pt}     
        \setlength{\fboxrule}{0.5pt}   
        \fbox{\includegraphics[scale=0.25]{Picture/login3acc.png}}
        \caption{Giao diện khi truy cập trang đăng nhập}
    \end{figure}
    
    \item Đăng nhập lần lượt các tài khoản để vào giao diện chat.\\
    \textbf{Kết quả: }Giao diện chat được hiển thị đầy đủ. Tên người dùng được đặt ở góc trái phía trên. Thanh điều hướng chia thành 2 phần (Channels và Peers) đại diện cho chat nhóm và chat cá nhân. Mặc định hệ thống sẽ vào nhóm chat \#general.
    \begin{figure}[H]
        \centering
        \setlength{\fboxsep}{2pt}     
        \setlength{\fboxrule}{0.5pt}   
        \fbox{\includegraphics[scale=0.25]{Picture/chatui.png}}
        \caption{Giao diện sau khi đăng nhập thành công sẵn sàng chat}
    \end{figure}
\end{itemize}

\subsubsection*{Bước 4: Kiểm thử chat trên trình duyệt}
\begin{itemize}
    \item \textbf{Test 1:} Alice, Bob, John chat cùng nhau ở \#general \\
    \textbf{Kết quả: }Nhóm \#general là nhóm mặc định tất cả các Peer khi đăng nhập, có vai trò là nhóm chat chung không có vai trò broadcast. Các tin nhắn được hiển thị trên giao diện của 3 người, có mốc thời gian gửi tin nhắn.
    \begin{figure}[H]
        \centering
        \setlength{\fboxsep}{2pt}     
        \setlength{\fboxrule}{0.5pt}   
        \fbox{\includegraphics[scale=0.25]{Picture/3general.png}}
        \caption{Giao diện 3 người chat ở nhóm \#general}
    \end{figure}
    \item \textbf{Test 2:} Alice, Bob, John chat cùng nhau ở \#broadcast \\
    \textbf{Kết quả: } Nhóm \#broadcast là nhóm mặc định tất cả các Peer khi đăng nhập, có vai trò là nhóm chat chung và có vai trò broadcast, các tin nhắn dạng broadcast sẽ được truyền đến bất kì nhóm chat nào, bất kì đoạn chat cá nhân nào.
        \begin{figure}[H]
        \centering
        \setlength{\fboxsep}{2pt}     
        \setlength{\fboxrule}{0.5pt}   
        \fbox{\includegraphics[scale=0.25]{Picture/3broadcast.png}}
        \caption{Giao diện 3 người chat ở nhóm \#broadcast}
    \end{figure}
    \item \textbf{Test 3:} Alice chat broadcast, Bob và John ở \#general (hoặc Bob chat broadcast, John chat broadcast và 2 người còn lại ở \#general)\\
    \textbf{Kết quả: } Khi Bob và John đang chat với nhau ở \#general khi này Alice nhắn tin broadcast thì Bob và John đều đồng thời nhận được tin nhắn broadcast.
    \begin{figure}[H]
        \centering
        \setlength{\fboxsep}{2pt}     
        \setlength{\fboxrule}{0.5pt}   
        \fbox{\includegraphics[scale=0.25]{Picture/1broad2gen.png}}
        \caption{Giao diện Bob và John ở \#general, Alice broadcast}
    \end{figure}
    \item \textbf{Test 4:} Alice chat riêng với Bob (hoặc Alice chat riêng với John, Bob chat riêng với John)\\
    \textbf{Kết quả: } Alice và Bob đang nhắn tin riêng
    \begin{figure}[H]
        \centering
        \setlength{\fboxsep}{2pt}     
        \setlength{\fboxrule}{0.5pt}   
        \fbox{\includegraphics[scale=0.25]{Picture/alicebob.png}}
        \caption{Giao diện Alice và Bob chat riêng}
    \end{figure}
    \item \textbf{Test 5:} John chat broadcast, Alice chat riêng với Bob (hoặc Bob chat broadcast, Alice chat broadcast và 2 người còn lại chat riêng với nhau)\\
    \textbf{Kết quả: } Alice và Bob đang nhắn tin riêng, khi này John nhắn tin broadcast thì Alice và Bob đều đồng thời nhận được tin nhắn broadcast.
    \begin{figure}[H]
        \centering
        \setlength{\fboxsep}{2pt}     
        \setlength{\fboxrule}{0.5pt}   
        \fbox{\includegraphics[scale=0.25]{Picture/johnbroadtoprivate.png}}
        \caption{Giao diện Alice và Bob chat riêng, John broadcast}
    \end{figure}
    \item \textbf{Test 6:} Tạo channel mới, cả 3 đều tham gia chat.\\
    \textbf{Kết quả: } Alice tạo channel \#CN04, Bob và John đều tham gia chat.
    \begin{figure}[H]
        \centering
        \setlength{\fboxsep}{2pt}     
        \setlength{\fboxrule}{0.5pt}   
        \fbox{\includegraphics[scale=0.25]{Picture/CN04.png}}
        \caption{Giao diện tạo nhóm mới và chat}
    \end{figure}
\end{itemize}

\subsubsection*{Kết luận:}  
Toàn bộ luồng hoạt động của ứng dụng chat hybrid — bao gồm khám phá, giao tiếp P2P, chat nhóm, broadcast và duy trì session — đều hoạt động ổn định, đúng với yêu cầu của đề tài. Hệ thống chứng minh tính tích hợp hoàn chỉnh giữa các thành phần HTTP, WebSocket, và P2P.

% -------------------------------------------------------------
\subsection*{5.2.4. Kiểm thử Ứng dụng Chat Hybrid bằng phần mềm Postman}
Sử dụng ứng dụng Postman Desktop để kiểm tra các API.
\subsubsection*{Bước 1: Khởi chạy Tracker Server và các Peer Clients}
\begin{itemize}
    \item \textbf{Mở Terminal 1: }Khởi chạy Tracker Server:
    \begin{lstlisting}[language=bash]
    python start_chatapp.py --server-ip 0.0.0.0 --server-port 8001
    \end{lstlisting}
    \textbf{Kết quả:} Server khởi động thành công và lắng nghe trên port 8001.
    \begin{lstlisting}[language=bash]
    ============================================================
    Starting Chat Tracker Server
    IP: 0.0.0.0
    Port: 8001
    ============================================================
    [Backend] Listening on port 8001
    \end{lstlisting}
\end{itemize}

\subsubsection*{Bước 2: Kiểm tra bằng cách gọi API}
\begin{itemize}
    \item \textbf{Login: }
    \begin{itemize}
        \item \textbf{Phương thức: }POST
        \item \textbf{URL: }http://127.0.0.1:8001/login
        \item \textbf{Body tab: }chọn \texttt{raw}, \texttt{JSON} và điền nội dung
        \begin{lstlisting}
        {
            "username": "admin",
            "password": "password"
        } 
        \end{lstlisting}
        \item Nhấn \textbf{Send}
        \item \textbf{Kết quả: }Status code \texttt{200 OK}
        \begin{lstlisting}
        {
            "status": "success",
            "message": "Login successful",
            "username": "admin",
            "token": "token_admin"
        }
        \end{lstlisting}
        \begin{figure}[H]
            \centering
            \setlength{\fboxsep}{2pt}
            \setlength{\fboxrule}{0.5pt}
            \fbox{\includegraphics[scale=0.5]{Picture/hcmut.png}}
            \caption{Giao diện gửi yêu cầu \texttt{/login}}
        \end{figure}
    \end{itemize}

    \item \textbf{Register User: }
    \begin{itemize}
        \item \textbf{Phương thức: }POST
        \item \textbf{URL: }http://127.0.0.1:8001/register
        \item \textbf{Body tab: }chọn \texttt{raw}, \texttt{JSON} và điền nội dung
        \begin{lstlisting}
        {
            "username": "alice",
            "password": "alice123"
        }
        \end{lstlisting}
        \item Nhấn \textbf{Send}
        \item \textbf{Kết quả: }Status code \texttt{200 OK}
        \begin{lstlisting}
        {
            "status": "success",
            "message": "User registered successfully",
            "username": "alice"
        }
        \end{lstlisting}
        \begin{figure}[H]
            \centering
            \setlength{\fboxsep}{2pt}
            \setlength{\fboxrule}{0.5pt}
            \fbox{\includegraphics[scale=0.5]{Picture/hcmut.png}}
            \caption{Giao diện gửi yêu cầu \texttt{/register}}
        \end{figure}
    \end{itemize}

    \item \textbf{Peer Registration (submit-info): }
    \begin{itemize}
        \item \textbf{Phương thức: }POST
        \item \textbf{URL: }http://127.0.0.1:8001/submit-info
        \item \textbf{Body tab: }chọn \texttt{raw}, \texttt{JSON} và điền nội dung
        \begin{lstlisting}
        {
            "username": "alice",
            "ip": "127.0.0.1",
            "port": 9101,
            "channels": ["general"]
        }
        \end{lstlisting}
        \item Nhấn \textbf{Send}
        \item \textbf{Kết quả: }Status code \texttt{200 OK}
        \begin{lstlisting}
        {
            "status": "success",
            "message": "Peer registered successfully",
            "peer_id": 0,
            "total_peers": 1
        }
        \end{lstlisting}
        \begin{figure}[H]
            \centering
            \fbox{\includegraphics[scale=0.5]{Picture/hcmut.png}}
            \caption{Đăng ký thông tin Peer qua endpoint \texttt{/submit-info}}
        \end{figure}
    \end{itemize}

    \item \textbf{Join Channel (add-list): }
    \begin{itemize}
        \item \textbf{Phương thức: }POST
        \item \textbf{URL: }http://127.0.0.1:8001/add-list
        \item \textbf{Body tab: }chọn \texttt{raw}, \texttt{JSON} và điền nội dung
        \begin{lstlisting}
        {
            "username": "alice",
            "channel": "tech-talk"
        }
        \end{lstlisting}
        \item Nhấn \textbf{Send}
        \item \textbf{Kết quả: }Status code \texttt{200 OK}
        \begin{lstlisting}
        {
            "status": "success",
            "message": "User added to channel successfully",
            "channel": "tech-talk",
            "members": ["alice"],
            "member_count": 1
        }
        \end{lstlisting}
        \begin{figure}[H]
            \centering
            \fbox{\includegraphics[scale=0.5]{Picture/hcmut.png}}
            \caption{Người dùng tham gia kênh qua endpoint \texttt{/add-list}}
        \end{figure}
    \end{itemize}

    \item \textbf{Get Peer List: }
    \begin{itemize}
        \item \textbf{Phương thức: }POST
        \item \textbf{URL: }http://127.0.0.1:8001/get-list
        \item \textbf{Body tab: }chọn \texttt{raw}, \texttt{JSON} và điền nội dung
        \begin{lstlisting}
        {}
        \end{lstlisting}
        \item Nhấn \textbf{Send}
        \item \textbf{Kết quả: }Status code \texttt{200 OK}
        \begin{lstlisting}
        {
            "status": "success",
            "peers": [
                {"username": "alice", "ip": "127.0.0.1", "port": 9101}
            ],
            "channels": {
                "tech-talk": ["alice"]
            }
        }
        \end{lstlisting}
        \begin{figure}[H]
            \centering
            \fbox{\includegraphics[scale=0.5]{Picture/hcmut.png}}
            \caption{Xem danh sách Peer hiện có qua endpoint \texttt{/get-list}}
        \end{figure}
    \end{itemize}

    \item \textbf{Server Status: }
    \begin{itemize}
        \item \textbf{Phương thức: }GET
        \item \textbf{URL: }http://127.0.0.1:8001/status
        \item \textbf{Body tab: }none
        \item Nhấn \textbf{Send}
        \item \textbf{Kết quả: }Status code \texttt{200 OK}
        \begin{lstlisting}
        {
            "status": "online",
            "stats": {
                "active_peers": 1,
                "active_channels": 1
            }
        }
        \end{lstlisting}
        \begin{figure}[H]
            \centering
            \fbox{\includegraphics[scale=0.5]{Picture/hcmut.png}}
            \caption{Kiểm tra trạng thái server qua endpoint \texttt{/status}}
        \end{figure}
    \end{itemize}

\end{itemize}
